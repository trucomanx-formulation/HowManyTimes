\documentclass[journal]{IEEEtran}

%\usepackage[brazil]{babel}
%\usepackage[T1]{fontenc}

\usepackage{theorem}        %%Lo agregue yo <========================================
\usepackage{algorithmic}        %%Lo agregue yo <========================================

\setcounter{secnumdepth}{7}

\newtheorem{theorem}{Theorem}%[section]
%\newtheorem{acknowledgement}[theorem]{Acknowledgement}
%\newtheorem{algorithm}[theorem]{Algorithm}
%\newtheorem{axiom}[theorem]{Axiom}
%\newtheorem{case}[theorem]{Case}
%\newtheorem{claim}[theorem]{Claim}
%\newtheorem{conclusion}[theorem]{Conclusion}
%\newtheorem{condition}[theorem]{Condition}
%\newtheorem{conjecture}[theorem]{Conjecture}
%\newtheorem{criterion}[theorem]{Criterion}
%\newtheorem{exercise}[theorem]{Exercise}
%\newtheorem{notation}[theorem]{Notation}
%\newtheorem{problem}[theorem]{Problem}
%\newtheorem{proposition}[theorem]{Proposition}
%\newtheorem{remark}[theorem]{Remark}
%\newtheorem{solution}[theorem]{Solution}
%\newtheorem{summary}[theorem]{Summary}

\newtheorem{definition}[theorem]{Definition}
\newtheorem{example}[theorem]{Example}
\newtheorem{lemma}[theorem]{Lemma}
\newenvironment{proof}[1][Proof]{\textbf{#1.} }{\ \rule{0.5em}{0.5em}}
\newtheorem{corollary}[theorem]{Corollary}
\newenvironment{algorithm}[1][Algorithm]{\textbf{#1.} }{}

\usepackage{amssymb}
\usepackage{graphicx}
\usepackage{amsmath}
\usepackage{psfrag}

\usepackage{accents}
%\usepackage[none]{hyphenat}

\usepackage[usenames,dvipsnames,svgnames,table]{xcolor}

\hyphenation{bet-ween re-pre-sen-ting} %
\sloppy

\begin{document}

\title{How many times do you need to roll a discrete random variable to get your statistic }


\author{Fernando P. Rivera 
\thanks{Manuscript received XXXX XX, 2014; revised XXXXX XX, 2014.}
\thanks{---------- ---------- ---------- ---------- ---------- ---------- ---------- ---------- ---------- ---------- ---------- }%%%%Fernando P. Rivera is with Department of Communications, State University of Campinas, Campinas, SP, Brazil. Email:fpujaico@decom.fee.unicamp.br. }
\thanks{---------- ---------- ---------- ---------- ---------- ---------- ---------- ---------- ---------- ---------- ---------- }}%%%%Jaime Portugheis   is with Faculty of Technology       , State University of Campinas, Limeira , SP, Brazil. Email:jaime@ft.unicamp.br  .}}

\markboth{IEEE Communications Letters,~Vol.~X,
No.~XX,~XXXXX~XXX}{Shell \MakeLowercase{\textit{et al.}}: Bare
Demo of IEEEtran.cls for Journals}

% make the title area
\maketitle
%%%%%%%\IEEEpeerreviewmaketitle


\begin{abstract}
This paper proposes a method to get the method to know
how many times do you need to roll a random variable to get your statistic

\end{abstract}

\begin{keywords}
Joint Probability
\end{keywords}

\IEEEpeerreviewmaketitle
%%%%%%%%%%%%%%%%%%%%%%%%%%%%%%%%%%%%%%%%%%%%%%%%%%%%%%%%%%%%%%%%%%%%%%%%%%%%%%%%%%%%%%%
%%%%%%%%%%%%%%%%%%%%%%%%%%%%%%%%%%%%%%%%%%%%%%%%%%%%%%%%%%%%%%%%%%%%%%%%%%%%%%%%%%%%%%%
%%%%%%%%%%%%%%%%%%%%%%%%%%%%%%%%%%%%%%%%%%%%%%%%%%%%%%%%%%%%%%%%%%%%%%%%%%%%%%%%%%%%%%%
\section{Introduction}
\label{sec:Intro}
 This
%%%%%%%%%%%%%%%%%%%%%%%%%%%%%%%%%%%%%%%%%%%%%%%%%%%%%%%%%%%%%%%%%%%%%%%%%%%%%%
%%%%%%%%%%%%%%%%%%%%%%%%%%%%%%%%%%%%%%%%%%%%%%%%%%%%%%%%%%%%%%%%%%%%%%%%%%%%%%
%%%%%%%%%%%%%%%%%%%%%%%%%%%%%%%%%%%%%%%%%%%%%%%%%%%%%%%%%%%%%%%%%%%%%%%%%%%%%%
\section{Theoretical foundation} 
\label{sec:theoretical}

\subsection{Probability of to get a value $x_m$ in the $k-th$ try} 
\label{subsec:PT}
Given a discrete random variable $X$ with $N$ different possible values, $X \in \{$ 
$x_0,$ $x_1,$ $...,$ $x_{N-1}$ $\}$, where $Pr(X=x_m)=p_m$ $\{\forall~~m \in Z^+ | 0 \leq m < N\}$.
We define as $T$ the discrete random variable that represent
the probability of to get a value $x_m$ until the $k-th$ roll of a
random variable $X$.
\begin{equation}
 Pr(T=k)=\left(1- p_m\right)^{k-1} p_m
\end{equation}

\subsection{Probability of to get a value $x_m$ in any of the first $k-th$ tries} 
\label{subsec:PTall}
Using the data of section \ref{subsec:PT}
\begin{equation}
 Pr(S=k)=\sum_{l=1}^k Pr(T=l)=\sum_{l=1}^k {\left(1- p_m\right)^{l-1} p_m}
\end{equation}



%%%%%%%%%%%%%%%%%%%%%%%%%%%%%%%%%%%%%%%%%%%%%%%%%%%%%%%%%%%%%%%%%%%%%%%%%%%%%%
\section{Final Remarks and Conclusions} 
\label{sec:Conclusions}
In this letter, we considered

%%%%%%%%%%%%%%%%%%%%%%%%%%%%%%%%%%%%%%%%%%%%%%%%%%%%%%%%%%%%%%%%%%%%%%%%%%%%%%
%%%%%%%%%%%%%%%%%%%%%%%%%%%%%%%%%%%%%%%%%%%%%%%%%%%%%%%%%%%%%%%%%%%%%%%%%%%%%%
%%%%%%%%%%%%%%%%%%%%%%%%%%%%%%%%%%%%%%%%%%%%%%%%%%%%%%%%%%%%%%%%%%%%%%%%%%%%%%
\section*{Acknowledgment}


%%%%%%%%%%%%%%%%%%%%%%%%%%%%%%%%%%%%%%%%%%%%%%%%%%%%%%%%%%%%%%%%%%%%%%%%%%%%%%
%\appendix
\section{Appendix} \label{sec:Appendix}

%%%%%%%%%%%%%%%%%%%%%%%%%%%%%%%%%%%%%%%%%%%%%%%%%%%%%%%%%%%%%%%%%%%%%%%%%%%%%%
%%%%%%%%%%%%%%%%%%%%%%%%%%%%%%%%%%%%%%%%%%%%%%%%%%%%%%%%%%%%%%%%%%%%%%%%%%%%%%
%%%%%%%%%%%%%%%%%%%%%%%%%%%%%%%%%%%%%%%%%%%%%%%%%%%%%%%%%%%%%%%%%%%%%%%%%%%%%%
\begin{thebibliography}{99}

\bibitem{fernando} Pujaico, F.; Portugheis, J.,
``Optimal Rate for Joint Source-Channel Coding of Correlated Sources Over 
Orthogonal Channels,'' Communications Letters, 2014.

\end{thebibliography}

\end{document}


%%%%%%%%%%%%%%%%%%%%%%%%%%%%%%%%%%%%%%%%%%%%%%%%%%%%%%%%%%%%%%%%%%%%%%%%%%%%%%
%%%%%%%%%%%%%%%%%%%%%%%%%%%%%%%%%%%%%%%%%%%%%%%%%%%%%%%%%%%%%%%%%%%%%%%%%%%%%%
%%%%%%%%%%%%%%%%%%%%%%%%%%%%%%%%%%%%%%%%%%%%%%%%%%%%%%%%%%%%%%%%%%%%%%%%%%%%%%
%%%%%%%%%%%%%%%%%%%%%%%%%%%%%%%%%%%%%%%%%%%%%%%%%%%%%%%%%%%%%%%%%%%%%%%%%%%%%%
%%%%%%%%%%%%%%%%%%%%%%%%%%%%%%%%%%%%%%%%%%%%%%%%%%%%%%%%%%%%%%%%%%%%%%%%%%%%%%
%%%%%%%%%%%%%%%%%%%%%%%%%%%%%%%%%%%%%%%%%%%%%%%%%%%%%%%%%%%%%%%%%%%%%%%%%%%%%%
%%%%%%%%%%%%%%%%%%%%%%%%%%%%%%%%%%%%%%%%%%%%%%%%%%%%%%%%%%%%%%%%%%%%%%%%%%%%%%
%%%%%%%%%%%%%%%%%%%%%%%%%%%%%%%%%%%%%%%%%%%%%%%%%%%%%%%%%%%%%%%%%%%%%%%%%%%%%%
%%%%%%%%%%%%%%%%%%%%%%%%%%%%%%%%%%%%%%%%%%%%%%%%%%%%%%%%%%%%%%%%%%%%%%%%%%%%%%
%%%%%%%%%%%%%%%%%%%%%%%%%%%%%%%%%%%%%%%%%%%%%%%%%%%%%%%%%%%%%%%%%%%%%%%%%%%%%%
%%%%%%%%%%%%%%%%%%%%%%%%%%%%%%%%%%%%%%%%%%%%%%%%%%%%%%%%%%%%%%%%%%%%%%%%%%%%%%
%%%%%%%%%%%%%%%%%%%%%%%%%%%%%%%%%%%%%%%%%%%%%%%%%%%%%%%%%%%%%%%%%%%%%%%%%%%%%%

\grid
